\documentclass{article}
\usepackage[a4paper,margin=0.5in]{geometry}
\usepackage{enumitem}
\usepackage{minted}
\usepackage{hyperref}

\begin{document}
\large

\textbf{Source : \href{https://youtu.be/DtGrdB8wQ_8}{Makefiles: 95\% of what you need to know - Gabriel Parmer}}

\section{Introduction}
\begin{itemize}
    \item It a build system.
          \begin{enumerate}[label=\roman*)]
              \item Automate the compilation and linking of source files into \emph{exectutables}.
              \item Recompilation on only the changed portion of source code and the portion dependant on the changed code.
              \item Make maintaining the build system easier by avoiding redudant code in build system.
          \end{enumerate}
    \item Used by any compiled language.
          \begin{itemize}
              \item C
              \item C++
              \item Fortran
              \item Latex
          \end{itemize}
\end{itemize}

\subsection{Printing}
\begin{itemize}
    \item To print Text, one can use any of the following:
          \begin{minted}[breaklines, frame=lines]{make}
    $(info This is an info text)
    $(warning This is an warning text)
    $(error This is an error text)
    \end{minted}
    \item Printing an \mintinline{make}{$(error)} will make the execution stop.
\end{itemize}


\subsection{Variables}
\begin{itemize}
    \item Variables in MakeFile are defines as
          \begin{minted}[frame=lines]{make}
    VARNAME = VALUE
    \end{minted}
    \item They can be used with the help of dollar symbol (\$).
          \begin{minted}[frame=lines]{make}
    PIVALUE = 3.14
    $(info Checking the Variable $(PIVALUE)) 
    \end{minted}
\end{itemize}


\subsection{Build Rule}
\begin{itemize}
    \item Every rule starts with a target (which is going to be built) followed by colon (:) followed by the dependencies the target require.
    \item The next line may or may not contain commands to exectue to get the dependencies.
    \item The dependencies may be
          \begin{itemize}
              \item Files
              \item Other Build Rules
          \end{itemize}

\end{itemize}


\subsubsection{Build rule depending on files}
\begin{minted}[breaklines, frame=lines]{make}
my_rule1 : source.c test.c
    gcc -c source.c -o source.o
\end{minted}
\begin{itemize}
    \item In this case, the \emph{my\_rule1} build rule will run only when \emph{source.c} and \emph{test.c} are available.
    \item Even if \emph{test.c} is not used in the command, since it is present as dependencies, \emph{test.c} must be available for \emph{my\_rule} to be executed.
\end{itemize}


\subsubsection{Build rule depending on other build rules}
\begin{minted}[breaklines, frame=lines]{make}
my_rule1 : my_rule2

my_rule2 : my_rule3
    gcc source.o -o source.bin

my_rule3 : source.c
    gcc -c source.c -o soruce.o
\end{minted}
\begin{itemize}
    \item In this case, \emph{my\_rule1} depends on \emph{my\_rule2} which in turn depends on \emph{my\_rule3} which in turn depends on \emph{source.c}.
    \item Here, only \emph{my\_rule2} and \emph{my\_rule3} has commands and those executed, when their correspoding dependancies are met.
\end{itemize}

\subsection{Build Rule with \emph{wildcards}}
\begin{itemize}
    \item Here, \emph{wildcards} are denoted by \emph{\%}.
    \item \emph{\%} denotes to any text.
\end{itemize}
\begin{minted}[breaklines, frame=lines]{make}
%.o : %.c
\end{minted}
\begin{itemize}
    \item \textbf{The above build rule indicates, to build anytext.o, then the dependency is the sametext.c}.
    \item This can be used to build any \emph{*.o}.
\end{itemize}



\subsubsection{Make special varaibles}
\begin{itemize}
    \item Used inside the command below the build rule.
    \item \emph{@} takes the left hand side of build rule.
    \item \textasciicircum takes the right hand side of build rule.
          \begin{minted}[breaklines, frame=lines]{make}
    source.o : source.c
        gcc -c $^ -o $@
    \end{minted}
    \item The above lines are same as:
          \begin{minted}[breaklines, frame=lines]{make}
    source.o : source.c
        gcc -c source.c -o source.o
    \end{minted}
    \item \textbf{These special varaibles along with wildcards can be used to create a generic building of *.o files from *.c files}.
          \begin{minted}[breaklines, frame=lines]{make}
    %.o : %.c
        gcc -c $^ -o $@
    \end{minted}
    \item But, they can run independently without a target, hence
          \begin{minted}[breaklines, frame=lines]{make}
    OBJFILES = test.o source.o

    all : $(OBJFILES)
    
    %.o : %.c
        gcc -c $^ -o $@
    \end{minted}
    \item Above, for every object files needed by \emph{all} build rule, the corresponding object files are created using \emph{\%.o} build rule.
\end{itemize}


\subsection{Building}
\begin{itemize}
    \item Builing is done with \emph{make}.
    \item This will run the first build rule and not \emph{all} build rule.
    \item Only if the dependencies changes, the build rules changes and this is propagated to other build rules which uses the dependencies.
    \item \textbf{Generarlly, MakeFile doesn't look for program oriented dependencies,like header files.}
    \item \textbf{\emph{@} symbol can be used to not to print the command.}
\end{itemize}



\section{Simple MakeFile Example}
\quad A sample example MakeFile used for building code inside \emph{./Simple\_MakeFileSystem} is shown below:


\inputminted[linenos, breaklines, frame=lines]{make}{./Simple_MakeFileSystem/Makefile}

\subsection{Disadvantages}
\begin{enumerate}[label=\Roman*)]
    \item The \emph{make} command recompiles the code where there is a change in .c files as only those are the depeneded for creating .o files.
          \begin{itemize}
              \item But,when only the header content changes the command doesn't compile.
          \end{itemize}
    \item Also, the include directory is liminted to one.
    \item In this, we have to give the .c and .o files.
\end{enumerate}


\section{Solving 2.1 - I)}
\begin{itemize}
    \item The particular problem, of header file not being seen can be solved by generating files (.d) that encodes make rule for .h dependencies.
    \item This can be done by adding the flags to the GCC compiler
          \begin{minted}[frame=lines]{make}
    -MP -MD
    \end{minted}
    \item Now, this will create .d files for the .c files which has dependend over header files so as to be used by make build system.
    \item We also need to clean the .d files.
    \item Now, the new MakeFile is shown below:
\end{itemize}

\inputminted[linenos, breaklines, frame=lines]{make}{./Simple_MakeFileSystem_2_1_I/Makefile}

\section{Solving 2.1 - II)}
\begin{itemize}
    \item \emph{+=} is used to append values to a varaible.
    \item To solve the issue of multiple include directories, we use foreach loop whose syntax is:
          \begin{minted}[frame=lines]{make}
    $(foreach var, range, cmds)
    \end{minted}
    \item An example to print numbers from 1 to 4 using foreach loop:
          \begin{minted}[frame=lines]{make}
    range = 1 2 3 4
    $(foreach i, $(range), $(info $(i)))
    \end{minted}
    \item Using the above information, now the make file can be made to have multiple include directories:
\end{itemize}

\inputminted[linenos, breaklines, frame=lines]{make}{./Simple_MakeFileSystem_2_1_II/Makefile}


\section{Solving 2.1 - III)}
\begin{itemize}
    \item The problem of finding .c files in the directores can be solved intializing the C directories to look for.
    \item Then, using foreach loop and wildcards, the .c files can be found using
          \begin{minted}[frame=lines]{make}
    CDIRS += .
    CDIRS += ./src
    
    CFILES = $(foreach cdir, $(CDIRS), $(wildcard $(cdir)/*.c))

    $(info $(CFILES))
    \end{minted}
    \item This will produce the output as:
          \begin{framed}
              ./prog.c  ./src/mod2.c ./src/mod1.c
          \end{framed}
    \item The object files to be created for each of the .c files in the same location so we use patsubst to get the .o files.
          \begin{minted}[frame=lines]{make}
    OBJFILES = $(patsubst %.c, %.o, $(CFILES))

    $(info $(OBJFILES))
    \end{minted}
    \item This will produce the output as:
          \begin{framed}
              ./prog.o  ./src/mod2.o  ./src/mod1.o
          \end{framed}
    \item Now, the complete Make file will all the Disadvantages solved is given below:
\end{itemize}

\inputminted[linenos, breaklines, frame=lines]{make}{./Simple_MakeFileSystem_2_1_III/Makefile}

\end{document}